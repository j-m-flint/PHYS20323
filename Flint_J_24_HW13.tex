%%%%%%%%%%%%%%%%%%%%%%%%%%%%%%%%%%%%%%%%%%%%%%%%%%%%%%%%%%%%
%%%%%%%%%%%%%%%%%%%%%%%%%%%%%%%%%%%%%%%%%%%%%%%%%%%%%%%%%%%%
%%%%%%%%%%%%%%%%%%%%%%%%%%%%%%%%%%%%%%%%%%%%%%%%%%%%%%%%%%%%
%%%%%%%%%%%%%%%%%%%%%%%%%%%%%%%%%%%%%%%%%%%%%%%%%%%%%%%%%%%%
%%%%%%%%%%%%%%%%%%%%%%%%%%%%%%%%%%%%%%%%%%%%%%%%%%%%%%%%%%%%
\documentclass[12pt]{article}
\usepackage{epsfig}
\usepackage{times}
\usepackage{fancyhdr}
\usepackage{pslatex}
\usepackage{amsmath}
\usepackage{mathrsfs}
\usepackage[dvipsnames]{xcolor}
\usepackage[hidelinks]{hyperref}%renewcommand{\topfraction}{1.0}
\renewcommand{\topfraction}{1.0}
\renewcommand{\bottomfraction}{1.0}
\renewcommand{\textfraction}{0.0}
\setlength {\textwidth}{6.6in}
\hoffset=-1.0in
\oddsidemargin=1.00in
\marginparsep=0.0in
\marginparwidth=0.0in                                                                               
\setlength {\textheight}{9.0in}
\voffset=-1.00in
\topmargin=1.0in
\headheight=0.0in
\headsep=0.00in
\footskip=0.50in     
\setcounter{page}{31}

\pagestyle{fancy}
\fancyhf{} % Clear all header and footer fields
\fancyfoot[R]{\thepage} 
\fancyfoot[L]{Latex Example} 
\renewcommand{\headrulewidth}{0pt} % Remove header rule
\renewcommand{\footrulewidth}{0pt} % Remove footer rule


\begin{document}
\def\pos{\medskip\quad}
\def\subpos{\smallskip \qquad}
\newfont{\nice}{cmr12 scaled 1250}
\newfont{\name}{cmr12 scaled 1080}
\newfont{\swell}{cmbx12 scaled 800}

%%%%%%%%%%%%%%%%%%%%%%%%%%%%%%%%%%%%%%%%%%%%%%%%%%%%%%%%%%%%
%     DO NOT CHANGE ANYTHING ABOVE THIS LINE
%%%%%%%%%%%%%%%%%%%%%%%%%%%%%%%%%%%%%%%%%%%%%%%%%%%%%%%%%%%%
%     DO NOT CHANGE ANYTHING ABOVE THIS LINE
%%%%%%%%%%%%%%%%%%%%%%%%%%%%%%%%%%%%%%%%%%%%%%%%%%%%%%%%%%%%
%     DO NOT CHANGE ANYTHING ABOVE THIS LINE
%%%%%%%%%%%%%%%%%%%%%%%%%%%%%%%%%%%%%%%%%%%%%%%%%%%%%%%%%%%%

\begin{center}
{\large
PHYS 20323/60323: Fall 2024 - LaTeX Example
}\\
%%%%%%%%%%%%%%%%%%%%%%%%%%%%%%%%%%%%%%%%%%%%%%%%%%%%%%%%%%%%
\end{center}
%%%%%%%%%%%%%%%%%%%%%%%%%%%%%%%%%%%%%%%%%%%%%%%%%%%%%%%%%%%%
% Bullet Point & Numbered list - lists can also be nested as below
%%%%%%%%%%%%%%%%%%%%%%%%%%%%%%%%%%%%%%%%%%%%%%%%%%%%%%%%%%%%
\begin{enumerate}
\item An electron is found to be in the spin state (in the z-basis): 
$\chi = A  \begin{pmatrix}
3i \\
4
\end{pmatrix}$ 

\begin{itemize}
\item [(a)] (5 points) Determine the possible values of A such that the state is normalized.
\newline
\item [(b)] (5 points) Find the expectation values of the operators ${\color{red}S_x}, {\color{purple}S_y}, {\color{orange}S_z},$
and
$S^2$
\newline
\end{itemize}

The matrix representations in the z-basis for the components of electron spin operators are given by:

$ 
{\color{red} S_x = \frac{\hbar}{2} \begin{pmatrix}
0 & 1 \\
1 & 0 
\end{pmatrix}}
{\color{purple} ;\hspace{1cm} S_x = \frac{\hbar}{2} \begin{pmatrix}
0 & -i \\
i & 0 
\end{pmatrix}}
{\color{orange} ;\hspace{1cm} S_x = \frac{\hbar}{2} \begin{pmatrix}
1 & 0 \\
0 & -1 
\end{pmatrix}}
$

\item The average electrostatic field in the earth’s atmosphere in fair weather is approximately given:


\begin{center}

$
\hfill \vec{E} = E_0 (Ae^{-\alpha z} + Be^{-\beta z}) \hat{z}, \hfill (1)
$
\end{center}

where A, B, $\alpha$, $\beta$ are positive constants and $z$ is the height above the (locally flat) earth surface.
\begin{itemize}
\item [(a)] (5 points) Find the average charge density in the atmosphere as a function of height
\newline
\item [(b)] (5 points) Find the electric potential as a function height above the earth.
\end{itemize}
\item \textbf{The following questions refer to stars in the Table below.} 

Note: There may be multiple answers.

\begin{tabular}{|c|c|c|c|c|c|}
\hline
Name & Mass & Luminosity & Lifetime & Temperature & Radius \\
\hline
$\beta$ Cyg. & 1.3 $M_\odot$ & 3.5 $L_\odot$ &  &  &  \\
\hline
$\alpha$ Cen. & 1.0 $M_\odot$ &  &  &  & 1 $R_\odot$ \\
\hline
$\eta$ Car. & 60. $M_\odot$ & $10^6$ $L_\odot$ & $8.0 \times 10^5$ years &  &  \\
\hline
$\epsilon$ Eri. & 6.0 $M_\odot$ & $10^3$ $L_\odot$ &  & 20,000 K &  \\
\hline
$\delta$ Scu. & 2.0 $M_\odot$ &  & $5.0 \times 10^8$ years &  & 2 $R_\odot$ \\
\hline
$\gamma$ Del. & 0.7 $M_\odot$ &  & $4.5 \times 10^{10}$ years & 5000 K &  \\
\hline
\end{tabular}

\begin{itemize}
\item [(a)] (4 points) Which of these stars will produce a planetary nebula.
\newline
\item [(b)] (4 points) Elements heavier than Carbon will be produced in which stars.
\end{itemize}
\end{enumerate}

%%%%%%%%%%%%%%%%%%%%%%%%%%%%%%%%%%%%%%%%%%%%%%%%%%%%%%%%%%%%



\end{document}
